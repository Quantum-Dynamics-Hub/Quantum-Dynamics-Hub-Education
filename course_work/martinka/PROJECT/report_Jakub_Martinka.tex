\documentclass{article}
\usepackage[letterpaper,margin=1in]{geometry}
\usepackage{xcolor}
\usepackage{fancyhdr}
\usepackage{tgschola} 

\renewcommand{\baselinestretch}{1.5}
\usepackage{sidecap}
\usepackage{graphicx}
\usepackage{caption}
\usepackage{subcaption}


\sidecaptionvpos{figure}{t}


\pagestyle{fancy}
\fancyhf{}
\fancyhead[C]{%
  \footnotesize\sffamily}

\newcommand{\soptitle}{Report}
\newcommand{\yourname}{Jakub Martinka}
\newcommand{\youremail}{jakubmartinka@centrum.cz}

\newcommand{\statement}[1]{\par\medskip
  \underline{\textcolor{blue}{\textbf{#1:}}}\space
}

\usepackage[
  pdftitle={\yourname - \soptitle},
  pdfauthor={\yourname},
]{hyperref}

\begin{document}

\begin{center}\LARGE\soptitle\\
\large of \yourname\ \\Excited States and Nonadiabatic Dynamics CyberTraining Workshop 2021
\end{center}

\hrule
\vspace{1pt}
\hrule height 1pt

\bigskip
\vspace{1cm}
For my project, I chose vinyl chloride as a test molecule. Vinyl chloride is mainly used for the preparation of polyvinyl chloride (PVC). At first, I wanted to use vinyl bromide as a test molecule, but NEXMD does not support the Br atom yet. Because of that, I used a similar molecule - vinyl chloride. The reason for choosing this molecule was that I have already worked with a similar one. It is a small molecule, making it an ideal system for test trajectories because in the case of very accurate methods it is a good idea not to perform expensive calculations. I also chose this molecule because I will be working on machine-learning of spin-orbital couplings in the future and I will use vinyl bromide as a test molecule. I was interested in using various programs to study this molecule, especially I focused on NEXMD and NewtonX using DFTB+, I also tried QXMD but after several problems, I gave up. I followed the NEXMD outline in the standard way that the manual recommends, so I got, for example, \hyperref[nexmd]{Figure 1a} and the spectrum in \hyperref[nexmd]{Figure 1b}, but I did not further analyze obtained information. \vspace{.4cm}

\begin{figure}[h!]
    \centering
\begin{subfigure}{.45\textwidth}
    \includegraphics[width=\textwidth]{figures/traj_NEXMD.png}   
    \caption{Potential energy as a function of time}
\end{subfigure}
\begin{subfigure}{.45\textwidth}
    \includegraphics[width=\textwidth]{figures/spectrum_NEXMD.png}   
    \caption{Absorption spectra}
\end{subfigure}
\label{nexmd}
\caption{Results for NEXMD}
\end{figure}
\vspace{.4cm}
The main part of my project was the use of Newton X by FSSH-NAMD with DFTB+, for example for spectrum simulation. I followed the standard procedure, which is also described in the tutorial, I prepared the structure and optimized it, calculated the normal modes, and used it to create 500 initial conditions. I used the halorg kit, which I downloaded from the official dftb+ website. Using NewtonX, I created a spectrum (\hyperref[spec]{Figure 2a})that is similar in the peak position compared to the experimental spectrum \hyperref[spec]{Figure 2b}. Then I prepared the input for SH-NAMD. I selected the initial state according to the oscillator strenght as S$_2$. Then I started a total of 30 trajectories. Most of them terminated around 60 fs. I analyzed the trajectory using available statistical methods, the results of which I present below. The pictures show the potential energies of the ten states as a function of time. The blue triangles indicate the current state of the system at each time step. \hyperref[sps]{Figure 3} shows a fraction of trajectories and average adiabatic population in the second excited state as a function of time.\vspace{.4cm}

\begin{figure}[h!]
    \centering
\begin{subfigure}{.42\textwidth}
    \includegraphics[width=\textwidth]{figures/spectrum2.png}   
    \caption{Photoabsorption cross section (Å$^2$.molecule$^{-1}$) as a function of energy (eV).}
\end{subfigure}
\begin{subfigure}{.42\textwidth}
    \includegraphics[width=\textwidth]{figures/spectrum_exp.png}   
    \caption{Experimental spectra \tiny{(Limão-Vieira \textit{et. al.}, CP 2006)}}
\end{subfigure}
\label{spec}
\caption{Potential energies of some trajectories - ten states as a function of time. The blue triangles indicate the current state of the system at each time step.}
\end{figure}

\begin{figure}[h!]
    \centering
    \includegraphics[width=.80\textwidth]{figures/pop_new.png}   
    \caption{Fraction of trajectories and average adiabatic population in the second excited state as a function of time}
    \label{sps}
\end{figure}


\begin{figure}[h]
    \centering
    \begin{subfigure}{.42\textwidth}
        \includegraphics[width=\textwidth]{figures/traj36.png}
        \caption{TRAJ36}    
    \end{subfigure}
    \begin{subfigure}{.42\textwidth}
        \includegraphics[width=\textwidth]{figures/traj37.png}
        \caption{TRAJ37}
    \end{subfigure}
    \begin{subfigure}{.42\textwidth}
        \includegraphics[width=\textwidth]{figures/traj42.png}
        \caption{TRAJ42}    
    \end{subfigure}
    \begin{subfigure}{.42\textwidth}
        \includegraphics[width=\textwidth]{figures/traj45.png}
        \caption{TRAJ45}
    \end{subfigure}
        \begin{subfigure}{.42\textwidth}
        \includegraphics[width=\textwidth]{figures/traj45.png}
        \caption{TRAJ48}    
    \end{subfigure}
    \begin{subfigure}{.42\textwidth}
        \includegraphics[width=\textwidth]{figures/traj48.png}
        \caption{TRAJ49}
    \end{subfigure}
    \begin{subfigure}{.42\textwidth}
        \includegraphics[width=\textwidth]{figures/traj50.png}
        \caption{TRAJ50}    
    \end{subfigure}
    \begin{subfigure}{.42\textwidth}
        \includegraphics[width=\textwidth]{figures/traj52.png}
        \caption{TRAJ52}
    \end{subfigure}
        \begin{subfigure}{.42\textwidth}
        \includegraphics[width=\textwidth]{figures/traj57.png}
        \caption{TRAJ57}    
    \end{subfigure}
    \begin{subfigure}{.42\textwidth}
        \includegraphics[width=\textwidth]{figures/traj59.png}
        \caption{TRAJ59}
    \end{subfigure}
    \caption{Potential energies of some trajectories - ten states as a function of time. The blue triangles indicate the current state of the system at each time step.}
\end{figure}



\end{document}